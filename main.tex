\documentclass[12pt]{amsart}
\usepackage{amsmath}
\usepackage{amsthm}
\usepackage{amsfonts}
\usepackage{amssymb}
\usepackage{ebproof}
\usepackage[margin=1in]{geometry}
\usepackage{hyperref}
\hypersetup{
    colorlinks=true,
    linkcolor=blue
}

\theoremstyle{definition}
\newtheorem{theorem}{Theorem}[section]
\newtheorem{lemma}[theorem]{Lemma}
\newtheorem{definition}[theorem]{Definition}
\newtheorem{corollary}[theorem]{Corollary}
\newtheorem{proposition}[theorem]{Proposition}
\newtheorem{conjecture}[theorem]{Conjecture}
\newtheorem{remark}[theorem]{Remark}
\newtheorem{example}[theorem]{Example}
\newtheorem{problem}[theorem]{Problem}
\newtheorem{notation}[theorem]{Notation}
\newtheorem{question}[theorem]{Question}
\newtheorem{caution}[theorem]{Caution}

\begin{document}

\title{Homework 6}

\maketitle

\begin{enumerate}
	\item The division algorithm says that given two natural numbers $n$ and $m$ we can write 
		\begin{displaymath}
			n = qm + r
		\end{displaymath}
		for two integers $q,r$ with $0 \leq r < m$. 

		Write this as a formula in predicate logic. 
	\item One way to express the infinitude of prime numbers is: There exists a prime number and 
		for any prime number $p$ there exists another prime $q$ with $q \gneq p$. 

		Write this as a formula in predicate logic. 
	\item Fermat's Last Theorem says: for any $n > 2$ there are no solutions to 
		\begin{displaymath}
			a^n + b^n = c^n 
		\end{displaymath}
		with $a,b,c \in \mathbb{N}$ with $a > 0, b > 0,$ and $c>0$. 
		
		Write this as a formula in predicate logic. 
	% \begin{enumerate}
	% 	\item For any natural number $n$, there is a natural number $m$ such that $m > 5n$. 
	% 	\item If $n$ is divisible by $m^2$, then $n$ is divisible by $m$. 
	% 	\item For every prime number $p$, there is a composite (non-prime) number bigger than 
	% 		$p$
	% 	\item There exists a unique solution to $\sqrt{n} = 0$. 
	% \end{enumerate}
	\item If the following are provable, give a proof. If not, give a model that invalidates it. 
	\begin{enumerate}
		\item $\displaystyle{\forall x~ (A(x) \to B(x)) \to \forall x~ (\neg A(x) \land B(x))}$
		\item $\displaystyle{\exists x~y~ B(x,y) \to \exists z~ B(z,z)}$
		% \item $\displaystyle{\forall x~ \neg (A(x) \lor B(x) \to C(x)) \to \exists x~ \neg C(x) }$
		\item $\displaystyle{\forall x~ A(x) \land \exists y~ (A(y) \to B) \to B}$
	\end{enumerate}
	\item Consider the statement:
	\begin{center}
		For any natural number $n$, either $n^2$ or $n^2 - 1$ is divisible by $3$. 
	\end{center}
	State this as formula. How far can you get to a formal proof of this statement via 
	natural deduction and using the division algorithm? What other facts would help you 
	get to a proof?
\end{enumerate}

\end{document}
