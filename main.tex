\documentclass[12pt]{amsart}
\usepackage{amsmath}
\usepackage{amsthm}
\usepackage{amsfonts}
\usepackage{amssymb}
\usepackage{ebproof}
\usepackage[margin=1in]{geometry}
\usepackage{hyperref}
\hypersetup{
    colorlinks=true,
    linkcolor=blue
}

\theoremstyle{definition}
\newtheorem{theorem}{Theorem}[section]
\newtheorem{lemma}[theorem]{Lemma}
\newtheorem{definition}[theorem]{Definition}
\newtheorem{corollary}[theorem]{Corollary}
\newtheorem{proposition}[theorem]{Proposition}
\newtheorem{conjecture}[theorem]{Conjecture}
\newtheorem{remark}[theorem]{Remark}
\newtheorem{example}[theorem]{Example}
\newtheorem{problem}[theorem]{Problem}
\newtheorem{notation}[theorem]{Notation}
\newtheorem{question}[theorem]{Question}
\newtheorem{caution}[theorem]{Caution}

\begin{document}

\title{Homework}

\maketitle

\begin{enumerate}
	\item Write out all elements of the set $\mathcal P(\{0,1,2\})$. 
		
	\item Prove the following identity: 
	\begin{displaymath}
		(X \cup Y) \cap Z = (X \cap Z) \cup (Y \cap Z)
	\end{displaymath}

	\item Prove the following identity:
	\begin{displaymath}
		(X \cup Y)^c = X^c \cap Y^c
	\end{displaymath}

	\item A set is called \textit{finite} if it only has a finite number of elements. 
		For a finite set $X$, we denote by $|X|$ the number of the elements 
		in $X$. It is called the \textit{order} of $X$. Establish the following 
		theorem: if $X$ is a finite set and $Y \subseteq X$ then 
		\begin{displaymath}
			|Y| \leq |X| 
		\end{displaymath}
		Note that you should show that $Y$ is also finite so be able to 
		use $|Y|$. 

		The converse of this result says that if $|Y| > |X|$ then $Y \nsubseteq X$. 
	% 		
	%
	% \item Given a family of sets $\{ X_i \mid i \in I \}$, we define the product 
	% 	\begin{displaymath}
	% 		\prod_{i \in I} X_i := \{ (x_i)_{i \in I} \mid x_i \in X_i \} 
	% 	\end{displaymath}
	% 	For example, if $X_i = \mathbb{N}$ and $I = \mathbb{N}$, elements of 
	% 	$\prod_{i \in \mathbb{N}} \mathbb{N}$ are infinite sequences 
	% 	\begin{displaymath}
	% 		n_0, n_1, n_2, n_3, n_4, n_5, \ldots, n_j, \ldots
	% 	\end{displaymath}
	% 	with $n_j \in \mathbb{N}$.
	%
	% 	Suppose that for each $i \in I$, we know that $X_i \subseteq Y_i$. 
	% 	Prove that 
	% 	\begin{displaymath}
	% 		\prod_{i \in I} X_i \subseteq \prod_{i \in I} Y_i
	% 	\end{displaymath}

	\item Decide (with proof) if the following are true always, sometimes, or never. 
		\begin{enumerate}
			\item Let $X$ and $Y$ be sets such that $X \setminus Y 
				= \varnothing$. Then $X = Y$.
			\item Let $X$, $Y$, and $Z$ be sets such that $X \setminus Y = Z$ 
				and $Y \subseteq Z$. Then $X = Y \cup Z$.
			\item Let $X$ be a set. Then $\{\varnothing\} \in \mathcal P(X)$. 
		\end{enumerate}
		
\end{enumerate}

\end{document}
